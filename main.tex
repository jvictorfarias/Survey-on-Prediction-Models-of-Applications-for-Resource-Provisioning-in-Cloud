%%%%%%%%%%%%%%%%%%%%%%%%%%%%%%%%%%%%%%%%%%%%%%%%%%%%%%%%%%%%%%%%%%%%%%
% How to use writeLaTeX: 
%
% You edit the source code here on the left, and the preview on the
% right shows you the result within a few seconds.
%
% Bookmark this page and share the URL with your co-authors. They can
% edit at the same time!
%
% You can upload figures, bibliographies, custom classes and
% styles using the files menu.
%
%%%%%%%%%%%%%%%%%%%%%%%%%%%%%%%%%%%%%%%%%%%%%%%%%%%%%%%%%%%%%%%%%%%%%%

\documentclass[12pt]{article}

\usepackage{sbc-template}

\usepackage{graphicx,url}

%\usepackage[brazil]{babel}   
\usepackage[utf8]{inputenc}  

     
\sloppy

\title{Survey on Prediction Models of Applications\\
 for Resource Provisioning in Cloud}
\author{Joao Victor Oliveira Farias}


\address{Universidade Federal do Ceará - Campus Quixadá\\
  Quixadá -- CE -- Brasil
\email{victorfarias.new@gmail.com}}
\begin{document} 

\maketitle

\begin{abstract}
  
\end{abstract}
     
\begin{resumo} 
  \cite{amiri2017survey}
  Este meta-artigo se caracteriza por um breve resumo explicativo sobre o artigo cujo tema é exposto no título do documento. O artigo busca apresentar um dos principais problemas de computação em nuvem, que é a predição de aprovisionamento de recursos para a escalabilidade dos seus serviços, bem como apresenta algumas maneiras de como minar esse problema. Um dos maiores agravantes da falta de predição de recursos, é a violação de SLA's(Service Level Agreements) que é um acordo entre o provedor de serviços e o cliente, o qual garante determinado nível de confiabilidade, nível este que deve ser garantido e mantido para que não haja quebra contratual entre as partes e os níveis de QoS sejam mantidos. Na busca de algoritmos de predição que evitem o desperdício de recursos, em sua maioria são usadas técnicas de machine learning que envolvem uma análise de histórico de uso de determinado serviço pelo cliente, e assumem uma possível previsão de futuro uso pelo mesmo. A conclusão do artigo é de que este já foi um grande problema no passado, mas hoje já existem muitas técnicas que ajudam, principalmente devido às novas técnicas de machine learning.
\end{resumo}


\bibliographystyle{sbc}
\bibliography{sbc-template}

\end{document}
